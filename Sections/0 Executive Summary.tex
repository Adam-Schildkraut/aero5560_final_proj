\section*{Executive Summary}
\addcontentsline{toc}{section}{Executive Summary}

% This report details a comprehensive analysis of the design, construction, and testing of the starboard side of a tailplane box. The TPB met the requirements for loading, and withstood a load of \textbf{1,177N}, which exceeded the limit load (863N), but just fell short of the ultimate load (1,324N). This was under the critical load case 1, which applied a vertical load of 880N to a Lug P on the outboard rib, with the inboard rib being secured in a jig. There was no obvious plastic deformation of the TPB after failure, as the failure mode was attributed to \textbf{rivet failure}, where Lug P partially ripped of the tip rib. 
% \\
% The design complies with FAA Aviation Regulation FAR 25.303 and FAR 25.625 where a factors of safety of 1.5 and 1.15 (for fittings) were considered. 
% \\

% The design was simplified from industry standard, however is compliant with the brief set out for this project. 
% \begin{itemize}
%     \item The layout of the box is simplified. It is reduced in size, and there is no aerofoil shape.
%     \item The loads are simplified. On a typical tailplane, there would normally be several hinge brackets and a distributed pressure load.
%     \item  There would also be many load cases, although these are often reduced to a few critical or envelope cases.
%     \item The methods of estimating costs are simpler.
%     \item The drawing requirements and systems are simplified.
% \end{itemize}


% The final TPB fit within the required dimensional requirements, and was 750mm long and 230mm wide, with a taper starting 150mm outboard and continuing to the tip.\\

% The total required cost for a completed ship set containing two of the design wings was calculated as  \textbf{\$4500.73}, and considers the costs of labour, drawings and materials. The calculated cost therefore exceeds the cost target of \$4200 by approximately 7 \%. Similarly the measured weight of the assembled TBP was recorded as \textbf{1566g} also exceeded the desired target weight of 1100g by 466g or 42\%.\\